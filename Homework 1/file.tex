\documentclass[12pt]{article}
\title{ECE 102 Homework 1}

\author{Lawrence Liu}
\usepackage{graphicx}
\usepackage{amsmath}
\setlength{\parindent}{0pt}

\begin{document}
\maketitle
\section*{Problem 1}
\subsection*{(a)}
$x(t-1)$ is simply the signal $x(t)$ delayed by $1$. So the plot would look like:
\begin{figure}[h]
\includegraphics[scale=0.4]{fig1a}
\centering
\end{figure}
\subsection*{(b)}
$x(4-t)$ is simply the signal delayed by $4$ and then reversed, so the plot would look like:
\begin{figure}[h]
\includegraphics[scale=0.4]{fig1b}
\centering
\end{figure}
\subsection*{(c)}
\begin{figure}[h]
\includegraphics[scale=0.4]{fig1c}
\centering
\end{figure}
\section*{Problem 2}
\subsection*{(a)}
\begin{align*}
a(t) &=\cos(\theta t)\sin(\psi t)\\
&=\frac{e^{i\theta t}+e^{-i\theta t}}{2}\frac{e^{i\psi t}-e^{-i\psi t}}{2i}\\
&=\frac{1}{4i}(e^{i(\theta+\psi) t}-e^{i(\theta-\psi)t}+e^{i(\psi-\theta)t}-e^{-i(\theta+\psi)t})\\
&=\frac{1}{4i}(e^{i(\theta+\psi) t}-^{-i(\theta+\psi)t})+e^{i(\psi-\theta)t}-e^{i(\theta-\psi)t})\\
&=\frac{2i}{4i}(\sin((\theta+\psi)t)+\sin((\psi-\theta)t))\\
&=\frac{1}{2}(\sin((\theta+\psi)t)-\sin((\theta-\psi)t))
\end{align*}
\subsection*{(b)}
Yes $a(t)$ can be periodic, for instance if we want it to have a period $3$ when $\theta=2\pi$ then $\psi=\frac{4\pi}{3}$.
\subsection*{(c)}
The period of $\cos(10t+1)$ is $\frac{\pi}{5}$ and the period of $\sin(4t-1)$ is $\frac{\pi}{2}$. The fundamental period of $x(t)$ is the least common multiple of these two periods, which is $\boxed{\pi}$.
\section*{Problem 3}
\subsection*{(a)}
The period of $\cos(3\Omega_0 t)$ is $\frac{2\pi}{3\Omega_0}$ and the period if $\cos(\Omega_0 t)$ is $\frac{2\pi}{\Omega_0}$. So the period of $x(t)=\cos(3\Omega_0 t)+5\cos(\Omega_0 t)$ is $\frac{2\pi}{\Omega_0}$. When $\Omega_0=\pi$ the period is $\boxed{2}$
\subsection*{(b)}
The power of a signal is 
$$P_x=\lim_{T\to\infty}\frac{1}{2T}\int_{-T}^{T}|x(t)|^2 dt$$
Since $x(t)$ is real we have
\begin{align*}
P_x&=\lim_{T\to\infty}\frac{1}{2T}\int_{-T}^{T} \left(\cos(3\Omega_0 t)+5\cos(\Omega_0 t)\right)^2dt\\
&=\lim_{T\to\infty}\frac{1}{2T}\int_{-T}\left(\cos(3\pi t)+5\cos(\pi t)\right)^2dt
\end{align*}

We know that the period of $x(t)$ is $2$ when $\Omega_0=\pi$. Thus let $T=2N$. Since $T\to\infty$ leads to $N\to\infty$ we get
$$
P_x=\lim_{N\to\infty}\frac{1}{4N}\int_{-2N}^{2N}(\cos(3\pi t)+5\cos(\pi t)^2dt
$$
Since $\left(\cos(3\pi t)+5\cos(\pi t)\right)^2$ is even:
$$P_x=\lim_{N\to\infty}\frac{1}{2N}\int_{0}^{2N}\left(\cos(3\pi t)+5\cos(\pi t)\right)^2dt$$
Since $\left(\cos(3\pi t)+5\cos(\pi t)\right)^2$ is periodic with period $2$
\begin{align*}
P_x&=\lim_{N\to\infty}\frac{1}{2}\int_{0}^{2}\left(\cos(3\pi t)+5\cos(\pi t)\right)^2dt\\
&=\lim_{N\to\infty}\frac{1}{2}\int_{0}^{2}\left(\cos^2(3\pi t)+10\cos(3\pi t)\cos(\pi t)+25\cos^2(\pi t)\right)^2dt
\end{align*}
Therefore we need to solve for $\int_{0}^{2}\cos^2(3\pi t)dt$, $\int_{0}^{2}\cos(3\pi t)\cos(\pi t)dt$ and $\int_{0}^{2}\cos^2(\pi t) dt$

\begin{align*}
\int_{0}^{2}\cos^2(3\pi t)dt&=3\int_{0}^{2/3}\cos^2(3\pi t)dt\\
&=3\int_{0}^{2/3}\frac{1+\cos(6\pi t)}{2}dt\\
&=1
\end{align*}

\begin{align*}
\int_{0}^{2}\cos(3\pi t)\cos(\pi t)dt&=\frac{1}{4}\int_{0}^{2}(e^{3\pi t}+e^{-3\pi t})(e^{\pi t}+e^{-\pi t})dt\\
&=\frac{1}{4}\left(\int_{0}^{2}(e^{4\pi t}+e^{-4\pi t})dt+\int_{0}^{2}(e^{2\pi t}+e^{-2\pi t})dt\right)\\
&=\frac{1}{2}\left(\int_{0}^{2}\cos(4\pi t)dt+\int_{0}^{2}\cos(2\pi t)dt\right)\\
&=0
\end{align*}
\begin{align*}
\int_{0}^{2}\cos^2(\pi t)dt&=\int_{0}^{2}\frac{1+\cos(2\pi t)}{2}dt\\
&=1
\end{align*}
Therefore we have that
\begin{align*}
P_x&=\frac{1}{2}(1+25)\\
&=\boxed{13}
\end{align*}
\subsection*{(c)}
\begin{align*}
P_1&=\lim_{T\to\infty}\frac{1}{2T}\int_{-T}^{T} \cos^2(3\Omega_0 t)dt\\
\end{align*}
Let $T=NT_0$ where $T_0=\frac{2\pi}{3\Omega_0}=\frac{2}{3}$ is the period of  $\cos(3\Omega_0 t)$. We have $T\to\infty$ leads to $N\to\infty$
\begin{align*}
P_1&=\lim_{N\to\infty}\frac{1}{2NT_0}\int_{-NT_0}^{NT_0} \cos^2(3\Omega_0 t)dt\\
&=\lim_{N\to\infty}\frac{1}{NT_0}\int_{0}^{NT_0} \cos^2(3\Omega_0 t)dt\\
&=\lim_{N\to\infty}\frac{1}{T_0}\int_{0}^{T_0} \cos^2(3\Omega_0 t)dt\\
&=\lim_{N\to\infty}\frac{1}{T_0}\int_{0}^{T_0} \frac{1+\cos(6\Omega_0 t)}{2}dt\\
&=\lim_{N\to\infty}\frac{1}{T_0}\frac{T_0}{2}\\
&=\frac{1}{2}
\end{align*}

We can do the same for $P_2$

\begin{align*}
P_2&=\lim_{T\to\infty}\frac{1}{2T}\int_{-T}^{T} 25\cos^2(\Omega_0 t)dt\\
\end{align*}
Let $T=NT_0$ where $T_0=\frac{2\pi}{\Omega_0}=2$ is the period of  $\cos(\Omega_0 t)$. We have $T\to\infty$ leads to $N\to\infty$
\begin{align*}
P_2&=25\lim_{N\to\infty}\frac{1}{2NT_0}\int_{-NT_0}^{NT_0} \cos^2(\Omega_0 t)dt\\
&=25\lim_{N\to\infty}\frac{1}{NT_0}\int_{0}^{NT_0} \cos^2(\Omega_0 t)dt\\
&=25\lim_{N\to\infty}\frac{1}{T_0}\int_{0}^{T_0} \cos^2(\Omega_0 t)dt\\
&=25\lim_{N\to\infty}\frac{1}{T_0}\int_{0}^{T_0} \frac{1+\cos(2\Omega_0 t)}{2}dt\\
&=25\lim_{N\to\infty}\frac{1}{T_0}\frac{T_0}{2}\\
&=\frac{25}{2}
\end{align*}

So therefore we get that indeed $P=P_1+P_2$

\subsection*{(d)}
No $\gamma(t)$ is not periodic. A signal made up of the summation of two signals are periodic if and only if $\frac{T_0}{T_1}=\frac{N}{M}$ where $T_0$ and $T_1$ are the periods of the two signals and $N$ and $M$ are integers. The period of $\cos(t)$ is $2\pi$ and the period of $\cos(\frac{\pi}{2}t)$ is $4$. It is impossible for $\frac{2\pi}{4}$ to be equal to one integer divided by another.


From the definition, the power of $\gamma(t)$ is:
$$P_{\gamma}=\lim_{T\to\infty}\frac{1}{2T}\int_{-T}^{T}|\gamma(t)|^2dt$$

However since $\gamma(t)$ isn't periodic, we cannot use the technique in the lecture where we rewrite $T$ as $NT_0$. Therefore the only way is to solve the integral and the limit:
\begin{align*}
P_\gamma&=\lim_{T\to\infty}\frac{1}{2T}\int_{-T}^{T}(\cos(t)+\cos(\frac{\pi}{2}t))^2dt\\
&=\lim_{T\to\infty}\frac{1}{2T}\int_{-T}^{T}(\cos^2(t)+\cos^2(\frac{\pi}{2}t)+2\cos(t)\cos(\frac{\pi}{2}t))dt\\
&=P_1+P_2+\lim_{T\to\infty}\frac{1}{2T}\int_{-T}^{T}2\cos(t)\cos(\frac{\pi}{2}t)dt
\end{align*}
Therefore, in order for $P_\gamma=P_1+P_2$ we need $\lim_{T\to\infty}\frac{1}{2T}\int_{-T}^{T}2\cos(t)\cos(\frac{\pi}{2}t)dt=0$. The first thing to note is that $\cos(t)\cos(\frac{\pi}{2}t)$ is even, therefore we have
$$\lim_{T\to\infty}\frac{1}{2T}\int_{-T}^{T}2\cos(t)\cos(\frac{\pi}{2}t)dt=\lim_{T\to\infty}\frac{1}{T}\int_{0}^{T}2\cos(t)\cos(\frac{\pi}{2}t)dt$$
Next we rewrite $\cos(t)\cos(\frac{\pi}{2}t)$ in terms of complex exponentials and simplify
\begin{align*}
\cos(t)\cos(\frac{\pi}{2}t)&=\frac{e^{it}+e^{-it}}{2}\frac{e^{i\frac{\pi}{2}t}+e^{-i\frac{\pi}{2}t}}{2}\\
&=\frac{1}{4}(e^{i(1+\frac{\pi}{2})t}+e^{-i(1+\frac{\pi}{2})t}+e^{i(\frac{\pi}{2}-1)t}+e^{-i(\frac{\pi}{2}-1)t})\\
&=\frac{1}{2}(\cos((1+\frac{\pi}{2})t)+\cos((\frac{\pi}{2}-1)t))
\end{align*}
Thus we get:
\begin{align*}
\lim_{T\to\infty}\frac{1}{T}\int_{0}^{T}2\cos(t)\cos(\frac{\pi}{2}t)dt&=\lim_{T\to\infty}\frac{1}{T}\int_{0}^{T}(\cos((1+\frac{\pi}{2})t)+\cos((\frac{\pi}{2}-1)t))dt\\
&=\lim_{T\to\infty}\frac{1}{T}\int_{0}^{T}\cos((1+\frac{\pi}{2})t)dt+\lim_{T\to\infty}\frac{1}{T}\int_{0}^{T}\cos((\frac{\pi}{2}-1)t)dt
\end{align*}

Therefore we will need to calculate $\lim_{T\to\infty}\frac{1}{T}\int_{0}^{T}\cos((1+\frac{\pi}{2})t)dt$ and $\lim_{T\to\infty}\frac{1}{T}\int_{0}^{T}\cos((\frac{\pi}{2}-1)t)dt$. Let us calculate $\lim_{T\to\infty}\frac{1}{T}\int_{0}^{T}\cos((1+\frac{\pi}{2})t)dt$ first by letting $T=NT_1$ where $T_1$ is the period of $\cos((1+\frac{\pi}{2})t)$. As $T\to\infty$, $N\to\infty$ therefore we get:
\begin{align*}
\lim_{T\to\infty}\frac{1}{T}\int_{0}^{T}\cos((1+\frac{\pi}{2})t)dt&=\lim_{N\to\infty}\frac{1}{NT_1}\int_{0}^{NT_1}\cos((1+\frac{\pi}{2})t)dt\\
&=\lim_{N\to\infty}\frac{1}{T_1}\int_{0}^{T_1}\cos((1+\frac{\pi}{2})t)dt\\
&=0
\end{align*}

Likewise for $\lim_{T\to\infty}\frac{1}{T}\int_{0}^{T}\cos((\frac{\pi}{2}-1)t)dt$, let $T=NT_2$ where $T_2$ is the period of $\cos((\frac{\pi}{2}-1)t)$. As $T\to\infty$, $N\to\infty$ therefore we get:
\begin{align*}
\lim_{T\to\infty}\frac{1}{T}\int_{0}^{T}\cos((\frac{\pi}{2}-1)t)dt&=\lim_{N\to\infty}\frac{1}{NT_2}\int_{0}^{NT_2}\cos((\frac{\pi}{2}-1)t)dt\\
&=\lim_{N\to\infty}\frac{1}{T_2}\int_{0}^{T_2}\cos((\frac{\pi}{2}-1)t)dt\\
&=0
\end{align*}
Thus we get $\lim_{T\to\infty}\frac{1}{2T}\int_{-T}^{T}2\cos(t)\cos(\frac{\pi}{2}t)dt=0$, and thus $P=P_1+P_2$
\section*{Problem 4}
\subsection*{(a)}
The even and odd parts of the signal are:
$$x_e(t)=\begin{cases} 
      |t| & |t|\leq 1 \\
      -\frac{1}{2} |t|+\frac{3}{2} & 1< |x|\leq 2 \\
      \frac{1}{2} & |t|>2 
   \end{cases}$$
$$x_o(t)=\begin{cases} 
      -\frac{1}{2} & t\leq -2 \\
      \frac{1}{2} t+\frac{1}{2} & -2< t\leq -1 \\
      0 & -1<t\leq1\\
      \frac{1}{2} t-\frac{1}{2}& 1< t\leq 2 \\
      \frac{1}{2}& 2<t \\
   \end{cases}$$
Plotted out, it looks like:
\begin{figure}[h]
\includegraphics[scale=0.4]{fig4a}
\centering
\end{figure}
\subsection*{(b)}
\begin{align*}
\int_{-\infty}^{\infty}|x(t)|^2dt&=\int_{-\infty}^{\infty}(x_o(t)+x_e(t))^2dt\\
&=\int_{-\infty}^{\infty}(x_o^2(t)+2x_e(t)x_o(t)+x_e^2(t))dt\\
&=\int_{-\infty}^{\infty}|x_o(t)|^2dt+\int_{-\infty}^{\infty}|x_e(t)|^2dt+\int_{-\infty}^{\infty}2x_e(t)x_o(t)dt
\end{align*}
$x_e(t)x_o(t)$ is odd. Therefore $\int_{-\infty}^{\infty}2x_e(t)x_o(t)dt=0$, and thus we get:
$$\int_{-\infty}^{\infty}|x(t)|^2dt=\int_{-\infty}^{\infty}|x_o(t)|^2dt+\int_{-\infty}^{\infty}|x_e(t)|^2dt$$
\end{document}